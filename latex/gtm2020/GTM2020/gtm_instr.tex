% This is ICM_instr.TEX the documentation file of
% the LaTeX2e class GTM2019
\documentclass[citeauthoryear]{GTM2020}
\pagestyle{plain}
\begin{document}

\newcounter{pp}
\setcounter{pp}{2}
\begin{center}
\LARGE\bfseries Instructions for Authors
\end{center}
\vskip20mm
\centerline{\Large\bf Table of Contents} \vskip10mm

\noindent\hbox to 5in {1. Introduction\dotfill \arabic{pp}}

\noindent\hbox to 5in {2. How to Invoke the GTM2020 Document
Class\dotfill \arabic{pp}}

\noindent\hbox to 126.7mm {3. The Heading of Your Contribution
\dotfill \arabic{pp}}
\addtocounter{pp}{2}

\noindent\hbox to 5in {4. How to Code Your Text \dotfill\arabic{pp}}

\noindent\hbox to 126.7mm {5. Predefined Theorem-like Environments\dotfill \arabic{pp}}
\addtocounter{pp}{1}

\noindent\hbox to 5in {6. Defining your Own Theorem-like Environments\dotfill \arabic{pp}}

\hbox to 5in {\qquad 6.1 \ Method 1 ({\itshape preferred\/})\dotfill \arabic{pp}}
\addtocounter{pp}{1}

\hbox to 5in {\qquad 6.2 \ Method 2 \dotfill \arabic{pp}}

\hbox to 5in {\qquad 6.3 \ Unnumbered Environments\dotfill \arabic{pp}}

\noindent\hbox to 5in {7. How to Edit Your Input (Source) File\dotfill \arabic{pp}}

\hbox to 5in {\qquad 7.1 \ Capitalization and Non-capitalization\dotfill \arabic{pp}}

\hbox to 5in {\qquad 7.2 \ Abbreviation of Words\dotfill \arabic{pp}}
\addtocounter{pp}{1}

\hbox to 5in {\qquad 7.3 \ Signs and Characters\dotfill \arabic{pp}}

\noindent\hbox to 126.7mm {8. Figures\dotfill \arabic{pp}}
\addtocounter{pp}{1}

\noindent\hbox to 5in {9. Tables\dotfill \arabic{pp}}

\noindent\hbox to 5in {10. References\dotfill \arabic{pp}}

\newpage
\section{Introduction}

Authors are requested to adhere strictly to these instructions;
{\em the class file must not be changed}.

The text output area is automatically set within an area of
5\,in horizontally  and 7.8\,in vertically.

If you are already familiar with \LaTeX{}, then the GTM2020 class
should not give you any major difficulties. The GTM2020 class is an
extension of the standard \LaTeX{} ``article'' document class.
Therefore you may use all ``article'' commands for the main body of
your paper to prepare your manuscript. It will change the layout to
the required GTM2020 style (it will for instance define the layout
of \verb|\section|). We had to invent some extra commands, which are
not provided by \LaTeX{} (e.\,g.\ \verb|\institute|, see also
Sect.\,\ref{contbegin})

Furthermore, the documentation provides suggestions about the proper
editing and use of the input files (capitalization, abbreviation etc.) (see
Sect.\,\ref{refedit} How to Edit Your Input File).

For the input of the references at the end of your contribution,
please follow our instructions given in Sect.\,\ref{refer} References.

\section{How to Invoke the GTM2020 Document Class}

GTM2020  class is invoked by replacing ``article'' by ``GTM2020'' in
the first line of your document:

\noindent \verb|\documentclass[citeauthoryear]{GTM2020}|

\begin{verbatim}
\begin{document}
  <Your paper>
\end{document}
\end{verbatim}
Option \verb|[citeauthoryear]| is necessary to cite properly references
in the text (see also Sect.\,\ref{refer} References).

\section{The Heading of Your Contribution}
\label{contbegin}

The title of a contribution should be coded as
follows:
\begin{verbatim}
\title{<Your contribution title>}
\end{verbatim}
All words in titles should be capitalized except for conjunctions,
prepositions (e.g.\ on, of, by, and, or, but, from, with, without,
under) and definite and indefinite articles (the, a, an) unless they
appear at the beginning. Formula letters must be typeset as in the text.
Titles have no end punctuation. If a long \verb|\title| must be divided
please use the code \verb|\\| (for new line).

Now the name(s) of the author(s) must be given:
\begin{verbatim}
\author{<author(s) name(s)>}
\end{verbatim}
If there is more than one author, the order is up to you;
the \verb|\and| command provides for the separation:
\begin{verbatim}
\author{Ivar Ekeland \and Roger Temam \and Alfred Holmes}
\end{verbatim}
Numbers referring to different addresses or affiliations are
to be attached to each author with the \verb|\inst{<no>}| command.
If you have done this correctly, this entry now reads, for example:
\begin{verbatim}
\author{Ivar Ekeland\inst{1} \and Roger Temam\inst{2}
    \and Alfred Holmes\inst{3}}
\end{verbatim}
The first name\footnote{Other initials are optional
and may be inserted if this is the usual
way of writing your name, e.\,g.\ Alfred J.~Holmes, E.~Henry Green.}
is followed by the surname.

Next the address(es) of institute(s), company etc.\ is (are) required.
If there is more than one address, the entries are numbered
automatically with \verb|\and|, in the order in which you type them.
Please make sure that the numbers match those placed next to
the authors' names to reflect the affiliation.
\begin{verbatim}
\institute{<name of an institute>
\and <name of the next institute>
\and <name of the next institute>}
\end{verbatim}

In addition, you can use
\begin{verbatim}
\email{<email address>}
\end{verbatim}
to provide your email address within \verb|\institute|. If you need to
typeset the tilde character --- e.\,g.\ for your web page in your unix
system's home directory --- the \verb|\homedir| command will happily do
this.

\medskip
If footnote like things are needed anywhere in the contribution heading
please code (immediately after the word where the footnote indicator
should be placed):
\begin{verbatim}
\thanks{<text>}
\end{verbatim}
\verb|\thanks| may only appear in \verb|\title|, \verb|\author|
and \verb|\institute| to footnote anything. If there are two or more
footnotes or affiliation marks to a specific item separate them with
\verb|\fnmsep| (i.\,e.\ {\itshape f}oot\emph note \emph mark
\emph{sep}arator).

\medskip
\noindent The command
\begin{verbatim}
\maketitle
\end{verbatim}
then formats the complete heading of your article. If you leave
it out the work done so far will produce \emph{no} text.

Then the abstract should follow. Simply code
\begin{verbatim}
\begin{abstract}
<Text of the summary of your article>
\end{abstract}
\end{verbatim}

\section{How to Code Your Text}
\label{headings}

All headings, as the contribution title, should be capitalized except
for conjunctions, prepositions (e.\,g.\ on, of, by, and, or, but,
from, with, without, under) and definite and indefinite articles (the,
a, an) unless they appear at the beginning. Formula letters must be
typeset as in the text.

Headings will be automatically numbered by the following codes.\\[2mm]
{\itshape Sample Input}
\begin{verbatim}
\section{This is a First-Order Title}
\subsection{This is a Second-Order Title}
\subsubsection{This is a Third-Order Title.}
\paragraph{This is a Fourth-Order Title.}
\end{verbatim}
{\it Note.} \verb|\section| and \verb|\subsection| have no end punctuation.
\verb|\subsubsection| and \verb|\paragraph| need to be punctuated at the end.

In addition to the above-mentioned headings your text may be structured
by subsections indicated by run-in headings (theorem-like environments).

\section{Predefined Theorem-like Environments}\label{builtintheo}

The following variety of run-in headings are at your disposal:
\begin{itemize}
\item
{\bfseries Bold} run-in headings with {\it italicized text\/}
as built-in environments:
\begin{verbatim}
\begin{corollary} <text> \end{corollary}
\begin{lemma} <text> \end{lemma}
\begin{proposition} <text> \end{proposition}
\begin{theorem} <text> \end{theorem}
\begin{axiom} <text> \end{axiom}
\end{verbatim}
\item
{\bfseries Bold} run-in headings with roman text as built-in environments:
\begin{verbatim}
\begin{definition} <text> \end{definition}
\begin{problem} <text> \end{problem}
\begin{property} <text> \end{property}
\begin{remark} <text> \end{remark}
\end{verbatim}
{\it Note.} The determinated notion in the text of a definition should be
{\it italicized}.\vskip2mm

\item
Further {\itshape italic} run-in headings with roman environment body may also
occur:
\begin{verbatim}
\begin{case} <text> \end{case}
\begin{conjecture} <text> \end{conjecture}
\begin{example} <text> \end{example}
\begin{exercise} <text> \end{exercise}
\begin{note} <text> \end{note}
\begin{question} <text> \end{question}
\begin{solution} <text> \end{solution}
\end{verbatim}
\item
The following generally appears as {\itshape italic} run-in heading:
\begin{verbatim}
\begin{proof} <text>    \qed    \end{proof}
\end{verbatim}
It is unnumbered and may contain an eye catching square (call for that
with \verb|\qed|) before the environment ends.
\end{itemize}

All the theorem-like environments are numbered automatically
throughout the sections of your document --- each with its own counter.

If you want the theorem-like environments to use the same counter
just specify the documentclass option \verb|envcountsame|:
\begin{verbatim}
\documentclass[citeauthoryear, envcountsame]{GTM2020}
\end{verbatim}
If your first call for a theorem-like environment then is e.\,g.\
\verb|\begin{lemma}|, it will be numbered 1; if corollary follows,
this will be numbered 2; if you then call lemma again, this will be
numbered 3.

But in case you want to reset such counters to 1 in each section,
please specify the documentclass option \verb|envcountreset|.

Even a numbering on section level (including the section counter) is
possible with the documentclass option \verb|envcountsect|.

\section{Defining your Own Theorem-like Environments}

We have enhanced the standard \verb|\newtheorem| command and slightly
changed its syntax to get two new commands \verb|\spnewtheorem| and
\verb|\spnewtheorem*| that now can be used to define additional
environments. They require two additional arguments namely the type
style in which the keyword of the environment appears and second the
style for the text of your new environment.

\verb|\spnewtheorem| can be used in two ways.

\subsection{Method 1 ({\itshape preferred})}

You may want to create an environment that shares its counter
with another environment, say {\em main theorem\/} to be numbered like
the predefined {\em theorem}. In this case, use the syntax
\begin{verbatim}
\spnewtheorem{<env_nam>}[<num_like>]{<caption>}
{<cap_font>}{<body_font>}
\end{verbatim}

\noindent
Here the environment with which the new environment should share its
counter is specified with the optional argument \verb|[<num_like>]|.
\smallskip

\noindent
{\em Sample Input}
\begin{verbatim}
\spnewtheorem{mainth}[theorem]{Main Theorem}{\bfseries}{\itshape}
\begin{theorem} The early bird gets the worm. \end{theorem}
\begin{mainth} The early worm gets eaten. \end{mainth}
\end{verbatim}

\noindent
{\em Sample Output}
\smallskip

\noindent {\bfseries Theorem 3.} {\em The early bird gets the worm.}
\smallskip

\noindent {\bfseries Main Theorem 4.} {\em The early worm gets eaten.}
\medskip

The sharing of the default counter (\verb|[theorem]|) is desired. If you
omit the optional second argument of \verb|\spnewtheorem| a separate
counter for your new environment is used throughout your document.

\subsection[Method 2]{Method 2 ({\itshape assumes {\tt[envcountsect]}
documentstyle option})}
\vskip3mm

\begin{verbatim}
\spnewtheorem{<env_nam>}{<caption>}[<within>]
{<cap_font>}{<body_font>}
\end{verbatim}

\noindent
This defines a new environment \verb|<env_nam>| which prints the caption
\verb|<caption>| in the font \verb|<cap_font>| and the text itself in
the font \verb|<body_font>|. The environment is numbered beginning anew
with every new sectioning element you specify with the optional
parameter \verb|<within>|.
\smallskip

\noindent
{\em Example} \leavevmode
\smallskip

\noindent
\verb|\spnewtheorem{joke}{Joke}[subsection]{\bfseries}{\rmfamily}|
\smallskip

\noindent defines a new environment called \verb|joke| which prints the
caption {\bf Joke} in boldface and the text in roman. The jokes are
numbered starting from 1 at the beginning of every subsection with the
number of the subsection preceding the number of the joke e.\,g.\ 7.2.1 for
the first joke in subsection 7.2.

\subsection{Unnumbered Environments}

If you wish to have an unnumbered environment, please
use the syntax
\begin{verbatim}
\spnewtheorem*{<env_nam>}{<caption>}{<cap_font>}{<body_font>}
\end{verbatim}

\section{How to Edit Your Input (Source) File}
\label{refedit}

\subsection{Capitalization and Non-capitalization}

\begin{itemize}
\item
The following should always be capitalized:
\begin{itemize}
\item
Headings (see Sect.\,\ref{headings})
\item
Abbreviations and expressions
in the text such as  Fig(s)., Table(s), Sect(s)., Chap(s).,
Theorem, Corollary, Definition etc. when used with numbers, e.\,g.\
Fig.\,3, Table\,1, Theorem 2.
\end{itemize}
\item
The following should {\em not\/} be capitalized:
\begin{itemize}
\item
The words figure(s), table(s), equation(s), theorem(s) in the text when
used without an accompanying number.
\item
Figure legends and table captions except for names and abbreviations.
\end{itemize}
\end{itemize}

\subsection{Abbreviation of Words}\label{abbrev}

\begin{itemize}
\item
The following {\em should\/} be abbreviated when they appear in running
text {\em unless\/} they come at the beginning of a sentence: Chap.,
Sect., Fig.; e.\,g.\ ``The results are depicted in Fig.\,5.''
``Figure 9 reveals that \dots .''\\
{\em Please note\/}: Equations should usually be referred to solely by
their number in parentheses: e.\,g.\ (14). However, when the reference
comes at the beginning of a sentence, the unabbreviated word
``Equation'' should be used: e.\,g.\ ``Equation (14) is very important.''
``However, (15) makes it clear that \dots .''
\item
If abbreviations of names or concepts are used
throughout the text, they should be defined at first occurrence,
e.\,g.\ Plurisubharmonic (PSH) Functions, Strong Optimization (SOPT)
Problem.
\end{itemize}

\subsection{Signs and Characters}
%
\subsubsection*{Special Signs.}
%
We have created further symbols for math mode (enclosed in \verb|\$|):
\begin{center}
\begin{tabular}{l@{\hspace{1em}yields\hspace{1em}}
c@{\hspace{3em}}l@{\hspace{1em}yields\hspace{1em}}c}
\verb|\grole| & $\grole$ & \verb|\getsto| & $\getsto$\\
\verb|\lid|   & $\lid$   & \verb|\gid|    & $\gid$
\end{tabular}
\end{center}
%
\subsubsection*{Gothic (Fraktur).}
%
In \LaTeX{} only the following gothic letters are available:
\verb|$\Re$| yields $\Re$ and \verb|$\Im$| yields $\Im$.
If gothic letters are {\it necessary}, please use
the amstex package of the American Mathematical Society (\verb|amsfonts|,
\verb|amssymb|).

For the real and the imaginary
parts of a complex number within math mode you should use:
\verb|$\mathrm{Re}$| (which yields $\mathrm{Re}$) or \verb|$\mathrm{Im}$|
(which yields $\mathrm{Im}$).
%
\subsubsection*{Special Roman.}
%
We created the blackboard bold characters listed below.
\begin{flushleft}
\begin{tabular}{@{}ll@{ yields }
c@{\hspace{1.em}}ll@{ yields }c}
\verb|\bbbc| & (complex numbers)   & $\bbbc$
  & \verb|\bbbf| & (blackboard bold F) & $\bbbf$\\
\verb|\bbbh| & (blackboard bold H) & $\bbbh$
  & \verb|\bbbk| & (blackboard bold K) & $\bbbk$\\
\verb|\bbbm| & (blackboard bold M) & $\bbbm$
  & \verb|\bbbn| & (natural numbers N) & $\bbbn$\\
\verb|\bbbp| & (blackboard bold P) & $\bbbp$
  & \verb|\bbbq| & (rational numbers)  & $\bbbq$\\
\verb|\bbbr| & (real numbers)      & $\bbbr$
  & \verb|\bbbs| & (blackboard bold S) & $\bbbs$\\
\verb|\bbbt| & (blackboard bold T) & $\bbbt$
  & \verb|\bbbz| & (whole numbers)     & $\bbbz$\\
\verb|\bbbone| & (symbol one)      & $\bbbone$
\end{tabular}
\end{flushleft}
\begin{displaymath}
\begin{array}{c}
\bbbc^{\bbbc^{\bbbc}} \otimes
\bbbf_{\bbbf_{\bbbf}} \otimes
\bbbh_{\bbbh_{\bbbh}} \otimes
\bbbk_{\bbbk_{\bbbk}} \otimes
\bbbm^{\bbbm^{\bbbm}} \otimes
\bbbn_{\bbbn_{\bbbn}} \otimes
\bbbp^{\bbbp^{\bbbp}}\\[2mm]
\otimes
\bbbq_{\bbbq_{\bbbq}} \otimes
\bbbr^{\bbbr^{\bbbr}} \otimes
\bbbs^{\bbbs_{\bbbs}} \otimes
\bbbt^{\bbbt^{\bbbt}} \otimes
\bbbz \otimes
\bbbone^{\bbbone_{\bbbone}}
\end{array}
\end{displaymath}

\subsubsection*{Vectors }{\bf(\verb|\vec{Symbol}|).}
Vectors may only appear in math mode. The default \LaTeX{} vector
symbol has been adapted to GTM2020 conventions.
\smallskip

\noindent \verb|$\vec{A \times B\cdot C}$| \ yields \ $\vec{A\times B\cdot C}$ \\
\verb|$\vec{A}^{T} \otimes \vec{B} \otimes \vec{\hat{D}}$| \ yields \
$\vec{A}^{T} \otimes \vec{B} \otimes \vec{\hat{D}}$

\section{Figures}

All schemas, graphs, diagrams and photographs are to be referred to
as figures. Do not use coloured photographs or figures.
A figure environment should be inserted in the text as close as
possible to the first reference to the figure.
Figures will be numbered automatically.
\smallskip

\noindent{\it Sample Input}

\begin{verbatim}
\begin{figure}
\vspace{2.0cm}
\caption{This is the caption of the figure displaying a white
eagle and a white horse on a snow field}
\end{figure}
\end{verbatim}

\noindent{\it Sample Output}

\begin{figure}
\vspace{2.0cm}
\caption{This is the caption of the figure displaying a white eagle and
a white horse on a snow field}
\end{figure}

\section{Tables}

Only horizontal lines should be used within a table, to distinguish
the column headings from the body of the table.
Table captions should be treated in the same way as figure legends,
except that the table captions appear {\it above\/} the tables and
are centred. The tables will be numbered automatically.
\smallskip

\noindent{\it Sample Input}

\begin{verbatim}
\begin{table}
\caption{Critical $N$ values}
\begin{tabular}{llllll}
\hline\noalign{\smallskip}
${\mathrm M}_\odot$ & $\beta_{0}$ & $T_{\mathrm c6}$ & $\gamma$
  & $N_{\mathrm{crit}}^{\mathrm L}$
  & $N_{\mathrm{crit}}^{\mathrm{Te}}$\\
\noalign{\smallskip}
\hline
\noalign{\smallskip}
 30 & 0.82 & 38.4 & 35.7 & 154 & 320 \\
 60 & 0.67 & 42.1 & 34.7 & 138 & 340 \\
120 & 0.52 & 45.1 & 34.0 & 124 & 370 \\
\hline
\end{tabular}
\end{table}
\end{verbatim}

\noindent{\it Sample Output}

\begin{table}
\caption{Critical $N$ values}
\begin{center}
\renewcommand{\arraystretch}{1.4}
\setlength\tabcolsep{3pt}
\begin{tabular}{llllll}
\hline\noalign{\smallskip}
${\mathrm M}_\odot$ & $\beta_{0}$ & $T_{\mathrm c6}$ & $\gamma$
  & $N_{\mathrm{crit}}^{\mathrm L}$
  & $N_{\mathrm{crit}}^{\mathrm{Te}}$\\
\noalign{\smallskip}
\hline
\noalign{\smallskip}
 30 & 0.82 & 38.4 & 35.7 & 154 & 320 \\
 60 & 0.67 & 42.1 & 34.7 & 138 & 340 \\
120 & 0.52 & 45.1 & 34.0 & 124 & 370 \\
\hline
\end{tabular}
\end{center}
\end{table}

\section{References}
\label{refer}

A reference list should be included at the end of your paper placing
the \LaTeX{} environment \verb|thebibliography| there. Only
essential references, which are directly referred to in the text,
should be included in the reference list. At the moment there is no
special Bib\TeX{} style for GTM2020 --- sorry. To use the reference
system you have to specify the option \verb|[citeauthoryear]| in the
\verb|documentclass|, like:
\begin{verbatim}
\documentclass[citeauthoryear]{GTM2020}
\end{verbatim}

References are cited in the text --- using the \verb|\cite| command ---
by name and year of publication in brackets, e.\,g.\  (\cite{back}),
according to your use of the \verb|\bibitem| command in the
\verb|thebibliography| environment. The coding is as follows:
if you choose your own label for the sources by giving an optional
argument to the \verb|\bibitem| command the citations
in the text are marked with the label you supplied.

\begin{verbatim}
The results in this section are a refined version
of (\cite{clar:eke}); the minimality result of Proposition~14
was the first of its kind.
\end{verbatim}
The above input produces the citation:
\smallskip

\noindent ``The results in this section are a refined version of
(Clarke and Ekeland, 1982); the min\-i\-mality \dots''.
\smallskip

\noindent Then the \verb|\bibitem| entry of the \verb|thebibliography|
environment should read:

\begin{verbatim}
\begin{thebibliography}{}  % (do not forget {})
.
\bibitem[Clarke and Ekeland, 1982]{clar:eke}
Clarke, F.\ and I.\ Ekeland (1982).
{\em Nonlinear oscillations and boundary-value problems for
Hamiltonian systems}. Arch.\ Rat.\ Mech.\ Anal., {\bf 78}, 315--333.
.
\end{thebibliography}
\end{verbatim}

For one author, use author's surname and the year (\cite{ab:45}).
For two authors, give both names and the year (\cite{ab:sm}).
For three or more authors, use a surname of the first author, plus
``et al.'', and the year (\cite{ab:et}, 1954).
You may put also the number of section, chapter, theorem, etc.,
after the year: (\cite{ab:45}, Theorem 2), (\cite{ab:sm}, Sect.\,2.1),
(\cite{back}, Chap.\,3). If giving a list of references, separate them
using semi-colons (Jones, 1965; \cite{tar:a}, b).
Put only the year in brackets when referring to the author(s) of
the referenced publication (for example, ``This work was first developed
by Clarke and Ekeland \cite{clar:eke}, and later expanded by \cite{subb},
who demonstrated that ...'')

The reference list should contain all citations occurring in the text,
ordered alphabetically by surname (with initials following). If there
are several works by the same author(s) the references should be listed
in the appropriate order indicated below:
\begin{itemize}
\item  One author: list works chronologically;
\item  Author and same co-author(s): list works chronologically;
\item  Author and different co-author(s): list works alphabetically
according to co-author(s).
\end{itemize}
If there are several works by the same author(s) and in the same year,
but which are cited separately, they should be distinguished by
the use of ``a'', ``b'', etc., put after the year.
\smallskip

References should appear {\it only\/} in the following three forms:
journal reference, reference to book, reference to multi-author work.
\smallskip

{\bf Journal reference\/} should include: author's surname and initials;
surnames and initials of remaining authors; year of publication
(in brackets); article title (in {\it italics\/}); abbreviated
journal title; volume number (in {\bf bold}, without ``Vol.'',
``No.'', etc.); page numbers (without ``pp.'');
language of publication except for English if relevant (in brackets) .
\smallskip

{\bf Reference to book\/} should include: author's surname and initials;
surnames and initials of remaining authors; year of publication
(in brackets); the book title (in {\it italics\/}); volume (with ``Vol.''),
chapter (with ``Chap.'')  or page (with ``pp.'') numbers if relevant;
the name of the publisher: place of publication;
language of publication except for English if relevant (in brackets).
\smallskip

{\bf Reference to multi-author work\/} should include after the year
of publication: the chapter title (in {\it italics\/});
``In:'' followed by book title; surname(s) and initials of editor(s)
and abbr.\ ``ed'' or ``eds'' (in brackets); ``Vol.'' and volume number when
appropriate; ``pp.'' and page numbers; the name of the publisher:
place of publication; language of publication except for English
if relevant (in brackets).
\smallskip

\noindent{\it Note.} Please, call your attention to punctuation marks
in the different forms of references (see {\it Samlpe\/} below).
\medskip

\noindent {\it Sample Input}
\begin{verbatim}
\begin{thebibliography}{}  % (do not forget {})

\bibitem[Abell, 1945]{ab:45}           % Journal reference
Abell, B.\,C.\ (1945). {\em The examination of cell nuclei}.
Biochemical Journal, {\bf 35}, 123--126.

\bibitem[Abell and Smith, 1956]{ab:sm}  % Journal reference
Abell, B.\,C.\ and S.\,E.\ Smith (1956). {\em Nucleic acid content
of microsomes}. Nature, {\bf 135(2)}, 7--9.

\bibitem[Abell et al.]{ab:et}     % Reference to multi-author work
Abell, B.\,C., R.\,C.\ Tagg and M.\ Push (1954).
{\em Enzyme catalyzed cellular transaminations}.
In: Advanced in Enzymology (Round, A.\,F., ed),
Vol.\,2, pp.\,125--247. Academic Press: New York.

\bibitem[Backer, 1963]{back}           % Reference to book
Backer, R.\,C.\ (1963). {\em Microscopic Staining Technicues},
Chap.\,3. Butterworth: London.

\bibitem[(1982)]{clar:eke}         % Journal reference
Clarke, F.\ and I.\ Ekeland (1982).
{\em Nonlinear oscillations and boundary-value problems for
Hamiltonian systems}. Arch.\ Rat.\ Mech.\ Anal., {\bf 78}, 315--333.

\bibitem[Subbotina (1986)]{subb}  % Reference to multi-author work
Subbotina, N.\,N.\  (1986). {\em Necessary and sufficient optimality
conditions for controls and trajectories}. In: Synthesis of optimal
control to game-theoretical problems (Subbotin, A.\,I.\  and
A.\,F.\ Kleimenov, eds), Vol.\,1, Chap.\,3, pp.\,86--96.
Inst.\ Math.\ Mech.: Sverdlovsk (in Russian).

\bibitem[Tarantello, 2001a]{tar:a}     % Journal reference
Tarantello, G.\ (2001a). {\em Subharmonic solutions with prescribed
minimalperiod for nonautonomous Hamiltonian systems}.
J.\ Diff.\ Eq.,  {\bf 2(3)}, 28--55.

\bibitem[Tarantello (2001b)]{tar:b}    % Journal reference
Tarantello, G.\ (2001b). {\em Subharmonic solutions for
Hamiltonian systems via a pseudoindex theory}. Annali di Matematica
Pura (to appear).

\end{thebibliography}
\end{verbatim}
\smallskip

\noindent{\it Sample Output}

\begin{thebibliography}{}  % (do not forget {})

\bibitem[Abell, 1945]{ab:45}
Abell, B.\,C.\ (1945). {\em The examination of cell nuclei}.
Biochemical Journal, {\bf 35}, 123--126.

\bibitem[Abell and Smith, 1956]{ab:sm}
Abell, B.\,C.\ and S.\,E.\ Smith (1956). {\em Nucleic acid content
of microsomes}. Nature, {\bf 135(2)}, 7--9.

\bibitem[Abell et al.]{ab:et}
Abell, B.\,C., R.\,C.\ Tagg and M.\ Push (1954). {\em Enzyme catalyzed
cellular transaminations}. In: Advanced in Enzymology (Round, A.\,F., ed),
Vol.\,2, pp.\,125--247. Academic Press: New York.

\bibitem[Backer, 1963]{back}
Backer, R.\,C.\ (1963). {\em Microscopic Staining Technicues},
Chap.\,3. Butterworth: London.

\bibitem[(1982)]{clar:eke}
Clarke, F.\ and I.\ Ekeland (1982).
{\em Nonlinear oscillations and boundary-value problems for
Hamiltonian systems}. Arch.\ Rat.\ Mech.\ Anal., {\bf 78}, 315--333.

\bibitem[Subbotina (1986)]{subb}
Subbotina, N.\,N.\  (1986). {\em Necessary and sufficient optimality
conditions for controls and trajectories}. In: Synthesis of optimal
control to game-theoretical problems (Subbotin, A.\,I.\  and
A.\,F.\ Kleimenov, eds), Vol.\,1, Chap.\,3, pp.\,86--96.
Inst.\ Math.\ Mech.: Sverdlovsk (in Russian).

\bibitem[Tarantello, 2001a]{tar:a}
Tarantello, G.\ (2001a). {\em Subharmonic solutions with prescribed
minimalperiod for nonautonomous Hamiltonian systems}.
J.\ Diff.\ Eq.,  {\bf 2(3)}, 28--55.

\bibitem[Tarantello (2001b)]{tar:b}
Tarantello, G.\ (2001b). {\em Subharmonic solutions for Hamiltonian
systems via a pseudoindex theory}. Annali di Matematica Pura (to appear).

\end{thebibliography}
\end{document}
