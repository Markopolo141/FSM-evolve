
\section{Introduction}
Biological species are well recognised as being engaged in an fight-for-survival and Evolutionary Game Theory has been used to analyse the strategies in such a fight.
Evolutionary Game Theory encompasses games of different forms, but one of the most standard forms concerns the continuous growth/decay of organism types where the organism types are defined by the strategy they play as they are continuously randomly paired to participate in a simultaneous symmetric two-player game where the expected payoff determines the growth-rate of their strategy.\cite{maynard,maynard2,weibull}

Whereas the sets of actions that actual organisms employ can depend-apon and influence the lifecycles of organisms within an ecosystem, which in turn determines their growth-rate.
These details, such as the stages of the lifecycles of organisms and the web of mutual influences between them, are seldom directly modelled in evolutionary game theory.
Lifecycles are most directly described by matrix models of population dynamics, where the growth-rate of an organism type is naturally given by its population matrix, and where the influences between lifecycle stages are recognised as a potential source of non-linear dynamics and chaotic behaviour \cite{doi:10.1080/10236198.2019.1699916, DEVRIES2020108875, population1}.

Within this paper we consider game-theoretic strategies in the context of lifecycles, and make a contribution by providing a proof that in such games that any search for equilibrium points need only consider pure strategies, and can ignore stochastic combinations of actions - or 'mixed' strategies.
In this way, dynamic points of interest in these complicated games can be discovered and characterised much more directly.

\subsection{Structure}
\begin{itemize}
\item In section \ref{sec:2} we introduce some of the concepts in population matrix models, particularly the concept of \textit{vital rates} and we illustrate how equilibrium population distributions are given by eigenvectors of the population matrix.
\item In section \ref{section:formalism} we give the elements of the game, showing how game-theoretic strategies and vital rates can be brought together to create population matrices for each strategy.
\item In section \ref{sec:equilibria} we give a verbal description of our main result about how equilibrium can be described by pure strategies alone.
\item In section \ref{appendix5} we give the mathematical demonstration of our primary result.
\item And section \ref{sec:conclusion} gives some concluding comments.
\end{itemize}


\section{Related Work}\label{sec:-1}

The formal theory of state has been introduced and considered in the context of population ecology by multiple authors, 
such as by Boling \cite{BOLING1973485} Caswell et.al \cite{nla.cat-vn662318} and Metz \cite{Metz1977}. Particularly Metz \& Diekmann\cite{nla.cat-vn2330051} considered the state of an individual organism (or the `i-state') as the information nessisary to make the response of an organism to its environment determinate, and contrasts this against the population state variable (or 'p-state') characterising the number of individuals in each i-state. 
Where it is possible to consider the population dynamics determined by the p-state, population matrix models have been increasingly employed in ecological studies to model the lifecycles and dynamics of various organisms between states.\cite{doi:10.1111/j.1461-0248.2010.01540.x}
Example models include modelling growth-rate by fecundity across age brackets \cite{leslie}, the dynamics of sexual reproduction and sex ratios \cite{Shyu2018,doi:10.1111/1365-2664.12177} as well as the dynamics of genetic spread and population control measures \cite{DEVRIES2020108875} etc.
In these matrix models, the ratio of population numbers moving between states are called \textit{vital rates}, which directly model the increase or decrease in organism numbers between states.

State also has a history as part of Game Theory literature, particularly the concept of state and actions are most directly modelled in Markov Decision Processes (MDPs) which can bear direct extension to many players in Lloyd Shapley's Stochastic-Games (SG) \cite{shapley53,Solan2015}. Stochastic Games are games where the game itself has a set of possible states or 'positions' (which might be the combination of the states of all players). Within this context each of the individual players have actions which they can execute which jointly determine the likely transitions between the states of the game and the immediate payoffs to each of them.

State has some history in the context of Evolutionary Game Theory, which can be seen in the context of various `evolutionary games on graphs' where the players have the state of belonging to nodes on a grid or graph structure. In these games the players at a node play actions against their nearest neighbours and it can be seen that the structure in these games capture a general sense of position as a state for the players, and this introduces unique and dynamic behaviours.\cite{nowak,spacial2,spacial4}

Another approach of integrating state into Evolutionary Game Theory extends from the works of Eitan Altman et.al \cite{markov2,markov3,markov4,markov5,markov8,markov9} who introduce the Markov-Decision-Evolutionary-Game (MDEG) and variants thereof.
In MDEG games each organism can occupy one of a finite set of states and has actions available to it depending on what state it is in.
The organisms are paired randomly and each of the participants chooses one of their available actions (as determined by their strategy) to execute.
The actions that the organisms execute determine their immediate payoffs and the probable transitions in state that they will make.
The expected sum of payoffs determine the growth-rate of the presence of the strategy in the population, and this then changes the composition of the population in which the random pairings occur.

Our work contrasts against other games, primarily as we model the growth-rate of strategies not in relation to the expected summation of payoffs, but in relation to the equilibrium between states defined by vital rates. This attitude is also reflected in the works of J. M. Cushing \cite{doi:10.1080/10236198.2016.1177522, Cushing2017,doi:10.1080/17513758.2019.1574034} who analyses evolutionary dynamics in population matrix models particularly focusing on bifurcation.

\section{Vital Rates and Population Matrix Models}\label{sec:2}

%It is also worth noting that fitness is not a very simple concept\cite{sep-fitness}
The demographic flow of individuals of a population between states of a lifecycle is sometimes described in ecological-studies by a matrix that is not necessarily Markov.
The simplest example of such matrices are Leslie Matrices used for studying the structure of populations of individuals transitioning between evenly spaced age-states.
Leslie Matrices are square, and they have form \cite{leslie}:

\begin{equation*}
\mathbf{M}=\begin{bmatrix}
    F_0 & F_1 & F_2 & \dots & F_{m-1} & F_m  \\
    P_0 &  0  &  0  & \dots &  0      &  0   \\
     0  & P_1 &  0  & \dots &  0      &  0   \\
     0  &  0  & P_2 & \dots &  0      &  0   \\
    \vdots & \vdots & \vdots & \ddots & \vdots & \vdots \\
     0  &  0  &  0  & \dots  & P_{m-1} &  0   \\
\end{bmatrix}
\end{equation*}

Where $P_i$ represents the probability of and individual in the $i$th age bracket successfully living into the $(i+1)$th age bracket, and $F_i$ is average number of offspring for an individual in $i$th age bracket within the duration of the age bracket.
The positive elements of these matrices identify the `flow' of individuals from one state to another and are called \textit{vital rates}.
For a column vector $\mathbf{n}(t)$ representing the number of individuals in each age-bracket at time $t$, $\mathbf{M}\mathbf{n}(t)$ gives the number of individuals in the population after the duration of one age bracket of time, and $\mathbf{M}^2\mathbf{n}(t)$ the number of individuals after two age brackets, and so on. $$\mathbf{n}(t+1)=\mathbf{M}\mathbf{n}(t)$$
Successive applications eventually yield a steady population profile and a constant exponential growth rate $\lambda$ given by the Euler–Lotka equation, where $\lambda$ is the dominant and only real-positive eigenvalue of the matrix, with the steady population profile $\mathbf{n}$ as its corresponding eigenvector, that is $\mathbf{M}\mathbf{n}=\lambda \mathbf{n}$.

A Leslie matrix is a specific example of a population/projection matrix which projects the growth/decline of a population whose vital rates remain constant.
However, more complicated scenarios exist where the vital rates vary depending on the distribution of the population itself.
For instance, the number of offspring and survival probability to a successive age-state may depend on the density of predators, mates and/or competitors for resources.
In this context the elements of $\mathbf{M}$ may depend on $\mathbf{n}(t)$ in some arbitrary way, which we denote as $\mathbf{M}_{\mathbf{n}(t)}$ and:
$$\mathbf{n}(t+1) = \mathbf{M}_{\mathbf{n}(t)}\mathbf{n}(t)$$
This is the form of a non-linear population matrix model; such models have been shown to yield dynamic behaviours such as non-linear growth, unstable cyclic behaviour and chaos.
Additionally as the matrix $\mathbf{M}$ is non-negative it has a maximum non-negative real eigenvalue $\lambda$ (which might be $1$) and corresponding population vector $\mathbf{n}$
$$\mathbf{M}_{\mathbf{n}}\mathbf{n} = \lambda\mathbf{n}$$
We consider then that the vector $\mathbf{n}$ characterises a potential equilibrium of the population with $\lambda$ its equilibrium growth rate. We now consider how to represent different strategies within the population.

