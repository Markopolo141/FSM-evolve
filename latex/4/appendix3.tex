\section{Matrix Proofs}\label{appendix5b}



\begin{Lemma}\label{lem2}
for a $n\times n$ matrix $A$, and $n$ column vector $b$, with $A^{b,k}$ denoting the matrix with its $k$th column as $b$.
If $\lambda$ is an eigenvalue for both $A$ and $A^{b,k}$ then it is also an eigenvalue for $\alpha A + (1-\alpha)A^{b,k}$ for any $\alpha \in \mathbb{R}$
\end{Lemma}
\begin{proof}
Consider the characteristic polynomials of $\lambda$ for $A$ and $A^{b,k}$:\\
$\det(A-\lambda I)=\det(A^{b,k}-\lambda I)=0$\\
If we let $C(\cdot)_{i,j}$ denote the $i$,$j$th cofactor of a matrix, then these determinants can be expanded along the $k$th column to give:\\
$\left(\sum_iA_{i,k}C(A-\lambda I)_{i,k}\right)-\lambda C(A-\lambda I)_{k,k}=\left(\sum_ib_iC(A-\lambda I)_{i,k}\right)-\lambda C(A-\lambda I)_{k,k}=0$\\
Therefore:\\
$\alpha\left(\left(\sum_iA_{i,k}C(A-\lambda I)_{i,k}\right)-\lambda C(A-\lambda I)_{k,k}\right) + (1-\alpha)\left(\left(\sum_ib_iC(A-\lambda I)_{i,k}\right)-\lambda C(A-\lambda I)_{k,k}\right)=0$\\
$=\left(\sum_i(\alpha A_{i,k}+(1-\alpha)b_i)C(A-\lambda I)_{i,k}\right)-\lambda C(A-\lambda I)_{k,k} =\det(\alpha A + (1-\alpha)A^{b,k}-\lambda I)=0$\\
Thus it is demonstrated that $\lambda$ is also an eigenvalue for $\alpha A + (1-\alpha)A^{b,k}$.
\end{proof}

\begin{Theorem}\label{th:2}
For a real $n\times n$ element-wise non-negative matrix $A$, and real element-wise non-negative column vector $b$, with $A^{b,k}$ denoting the matrix with its $k$th column as $b$.
For the matrix mapping $B(\alpha) = \alpha A + (1-\alpha)A^{b,k}$ defined on a range $0\le\alpha\le1$. If $\rho(B(\alpha))$ denotes the spectral radius of $B(\alpha)$.\\ Then $\rho(B(\alpha))$ is continuous, and either constant or strictly monotonic with $\alpha$.
\end{Theorem}
\begin{proof}
Because $B(\alpha) = \alpha A + (1-\alpha)A^{b,k}$ is a matrix continuous in all its elements it is thus well established that it will have $n$ continuous eigenvalues\cite{matrix1}.\footnote{An informal outline of the proof is that: 1. If the elements of a matrix are continuous 2. Then the coefficients of the characteristic polynomial are continuous (as they are additions and multiplications of them) 3. Then the roots of the characteristic polynomial are continuous (see \cite{roots1}) 4. Hence the eigenvalues are continuous}\\
It is thus straightforward to note that the function $\rho(B(\alpha))$ is also continuous with $\alpha$ for all $\alpha$.\\
Furthermore that the value $\rho(B(\alpha))$ is itself an eigenvalue of $B(\alpha)$ for all $\alpha$ via the Perron-Frobenius theorem\footnote{\label{note1}that an element-wise non-negative matrix has an eigenvalue equal to its spectral radius. The proof of this one seems quite difficult to find even though it certainly is stated in literature (for instance \cite{bo1}). Furthermore it also something which might be seem as intuitive, as any non-negative matrix can be constructed as the limit of a sequence of element-wise positive matricies and as the eigenvalues of a matrix must be continuous on their elements, thus that the corresponding property should also hold in the limit.}.
Suppose for a contradiction that $\rho(B(\alpha))$ is not monotone, in this case there must exist at least three values of alpha, $\alpha_1<\alpha_2<\alpha_3$ such that $\rho(B(\alpha_2))>\max(\rho(B(\alpha_0)),\rho(B(\alpha_3)))$ or $\rho(B(\alpha_2))<\min(\rho(B(\alpha_0)),\rho(B(\alpha_3)))$.
\begin{itemize}[leftmargin=*,labelsep=4mm]
\item	suppose that $\rho(B(\alpha_2))>\max(\rho(B(\alpha_1)),\rho(B(\alpha_3)))$: let $\beta$ be a value between $\rho(B(\alpha_2))$ and $\max(\rho(B(\alpha_1)),\rho(B(\alpha_3)))$. Thus via the intermediate value theorem there exists $\gamma_1$ ($\alpha_1<\gamma_1<\alpha_2$) and $\gamma_2$ ($\alpha_2<\gamma_2<\alpha_3$) such that $\rho(B(\gamma_1))=\rho(B(\gamma_2))=\beta$. Thus $\beta$ is an eigenvalue of $B(\alpha_1)$ (via Lemma \ref{lem2}), and $\beta > \rho(B(\alpha_1))$ which contradicts the construction of $\rho(B(\alpha_1))$.
\item   suppose that $\rho(B(\alpha_2))<\min(\rho(B(\alpha_1)),\rho(B(\alpha_3)))$: let $\beta$ be a value between $\rho(B(\alpha_2))$ and $\min(\rho(B(\alpha_1)),\rho(B(\alpha_3)))$. Thus via the intermediate value theorem there exists $\gamma_1$ ($\alpha_1<\gamma_1<\alpha_2$) and $\gamma_2$ ($\alpha_2<\gamma_2<\alpha_3$) such that $\rho(B(\gamma_1))=\rho(B(\gamma_2))=\beta$. Thus $\beta$ is an eigenvalue of $B(\alpha_2)$ (via Lemma \ref{lem2}), and $\beta > \rho(B(\alpha_2))$ which contradicts the construction of $\rho(B(\alpha_2))$.
\end{itemize}
Therefore $\rho(B(\alpha))$ is monotonic.\\
If there does not exist any $\alpha_1,\alpha_2\in[0,1]$ such that $\rho(B(\alpha_1))$ = $\rho(B(\alpha_2))$\\
\-\hspace{8mm}then $\rho(B(\alpha))$ is strictly monotonic.\\
If there does exist an $\alpha_1,\alpha_2\in[0,1]$ such that $\rho(B(\alpha_1))$ = $\rho(B(\alpha_2))$\\
\-\hspace{8mm}then $\rho(B(\alpha))$ is constant via lemma \ref{lem2}.\\
Which completes the proof.
\end{proof}


\begin{Theorem}\label{th:3}
For a $n\times n$ matrix $A_{i,j}$, and column vectors $b$ and $c$, with $A^{b,k}$ and $A^{c,k}$ denoting the matrix with its $k$th column as $b$ and $c$ respectively. For the matrix mapping $B(\alpha) = \alpha A^{b,k} + (1-\alpha)A^{c,k}$ for $\alpha\in\mathbb{R}$. Let $\lambda(\alpha)$ and $v(\alpha)$ be an eigenvalue/vector pairing of $B(\alpha)$\\
If there exists different $\alpha_1$ and $\alpha_2$ such that $\lambda(\alpha_1)=\lambda(\alpha_2)$, then $\lambda(\alpha)=\lambda(\alpha_1)$ and $v(\alpha)=\frac{\alpha-\alpha_1}{\alpha_2-\alpha_1}v(\alpha_2)+\frac{\alpha-\alpha_2}{\alpha_1-\alpha_2}v(\alpha_1)$ is an eigenvalue/vector pairing for all $\alpha$, with the $k$th value of the eigenvector - $v(\alpha)_k$ being constant.
\end{Theorem}
\begin{proof}
if $b=c$ then $A^{b,k}=A^{c,k}$ and $B(\alpha) = A^{b,k}$, then $\lambda(\alpha)=\lambda(\alpha_1)$ and $v(\alpha)=v(\alpha_1)$ is trivial solution which fulfills the proof. Otherwise $b\ne c$.\\
Since $\lambda(\alpha_1)$ is an eigenvalue for all $B(\alpha)$ via Lemma \ref{lem2} then $\frac{\partial \lambda(\alpha)}{\partial \alpha}=0$ and $\lambda(\alpha)=\lambda(\alpha_1)=\lambda$ is true.\\
As: $~\left(\alpha A^{b,k} + (1-\alpha)A^{c,k}\right)v(\alpha)=\lambda v(\alpha)$\\
If $v(\alpha)_k$ denotes the $k$th value of $v(\alpha)$, then differentiating with respect to $\alpha$ \\
Gives: $\left(\alpha A^{b,k} + (1-\alpha)A^{c,k} - \lambda I\right)\frac{\partial v(\alpha)}{\partial\alpha}+(b-c)v(\alpha)_k=0$\\
now, there are two cases:\\
if there is an $\alpha_3$ such that $v(\alpha_3)_k=0$ then:\\
\-\hspace{8mm}$\left(\alpha_3 A^{b,k} + (1-\alpha_3)A^{c,k} - \lambda I\right)\frac{\partial v(\alpha_3)}{\partial\alpha}=0$\\
\-\hspace{8mm}Setting $\frac{\partial v(\alpha)}{\partial\alpha}=0$ is permissible, making $v(\alpha)_k=0$.\\
\-\hspace{8mm}therefore constant $v(\alpha)=v(\alpha_3)$ is solution which fulfills the proof\\
if there is not an $\alpha_3$ such that $v(\alpha_3)_k=0$, then:\\
\-\hspace{8mm}It is possible do scaling, thus setting $v(\alpha)_k=d$ to be a non-zero constant and therefore $\frac{\partial v(\alpha)_k}{\partial \alpha}=0$\\
\-\hspace{8mm}Thus there is a $\frac{\partial v(\alpha)}{\partial\alpha}$ such that: $\left(\alpha A^{b,k} + (1-\alpha)A^{c,k} - \lambda I\right)\frac{\partial v(\alpha)}{\partial\alpha}+d(b-c)=0$\\
\-\hspace{8mm}Thus there is a $\frac{\partial v(\alpha)}{\partial\alpha}$ such that for all $i$: $\left(\sum_{j,j\ne k}\left(A_{i,j}-\lambda I_{i,j}\right)\frac{\partial v(\alpha)_j}{\partial\alpha}\right) +d(b_i-c_i)=0 $\\
\-\hspace{8mm}Therefore $\frac{\partial v(\alpha)}{\partial\alpha}$ can be constant\\
\-\hspace{8mm}Therefore $v(\alpha)=\frac{\alpha-\alpha_1}{\alpha_2-\alpha_1}v(\alpha_2)+\frac{\alpha-\alpha_2}{\alpha_1-\alpha_2}v(\alpha_1)$ is only linear solution that adjoins $v(\alpha_2)$ and $v(\alpha_2)$.\\
Which completes the proof.
\end{proof}

