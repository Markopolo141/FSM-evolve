\section{Description of the Game}\label{section:formalism}

Consider an ecosystem of different species of organisms, where the organisms of each species have a distinct set of states which they can occupy. Further imagine that each state has a set of actions which an organism in the state can execute.
Let:
\begin{itemize}[leftmargin=*,labelsep=4mm]
\item   $K$ be a finite set of species
\item	$S$ be the finite set of all states
\item   $S_k$ be nonempty disjoint subsets of states $S$ available to species $k\in K$
\item   $A$ be the finite set of all actions
\item   $A_k$ be nonempty disjoint subsets of actions $A$ available to species $k\in K$
\item   $A_{k,s}$ be nonempty subset of actions $A_k$ available to species $k\in K$ in state $s\in S_k$
\end{itemize}
Further imagine that each individual organism has a strategy in addition to its state, which encodes the probabilities of what action it will execute depending on the state it is in.
Let:

\begin{itemize}[leftmargin=*,labelsep=4mm]
\item   $W^k$ be the set of possible strategies for species $k\in K$, such that for any strategy $w^k \in W^k$ that $w^k_{a,s}$ denotes the probability that an organism with strategy $w^k$ will execute action $a\in A_{k,s}$ if it is in state $s\in S_k$.
\end{itemize}
The elements of $W^k$ are all that satisfy the basic rules of probability:
\begin{itemize}[leftmargin=*,labelsep=4mm]
\item[--]   probabilities of taking actions from any state must sum to one:\\\-\hspace{8mm} $\forall k\in K~~\forall w^k\in W^k~~\forall s\in S_k~~ \sum_{a\in A_{k,s}}w^k_{a,s}=1$
\item[--]   all probabilities must be non-negative:\\\-\hspace{8mm} $\forall k\in K~~\forall w^k\in W^k~~\forall s\in S_k~~\forall a\in A_{k,s}~~ w^k_{a,s}\ge 0$
\item[--]   inaccessible actions have probability of zero:\\\-\hspace{8mm} $\forall k\in K~~\forall w^k\in W^k~~\forall s\in S_k~~\forall a\notin A_{k,s}~~ w^k_{a,s}= 0$
\end{itemize}
The remaining elements are:

\begin{itemize}[leftmargin=*,labelsep=4mm]
\item	$P_{t,k,s,w}$ is the number\footnote{$P_{t,k,s,w}$ defines a distribution of the population at time $t$, which may be normalised and hence represent a probability distribution or left unnormalised as representing actual numbers of organisms. The only constraint is that it be non-negative $\forall t,k,s,w~~P_{t,k,s,w}\ge 0$. If the probability distribution is to be normalised then the normalisation can either be `built-in' to the transmission $T$ terms or included as a separate step in the algorithm \ref{al:1}} of organisms at time $t$ of species $k\in K$ in a state $s\in S^k$ with strategy $w\in W^k$; 
\item   $P^*_{t,k,s,a} = \sum_{w^k\in W^k}P_{t,k,s,w^k}w^k_{a,s}$ is the number of organisms at time $t$ of species $k\in K$ in a state $s\in S^k$ which are going to take action $a\in A_{k,s}$.
\item   $T_{k,s,a}(P^*_t)$ are positive functions of argument\footnote{where $P^*_t$ is shorthand for the set of all the numbers across $k$, $s$ and $a$,  $P^*_t = \{P^*_{t,k,s,a}~|~k\in K,~s\in S_k,~a\in A_{k,s}\}$} $P^*_t$, giving transmission of organisms (of a strategy $w^k$) \textit{to} state $s\in S_k$ when action $a\in A_k$ is executed by an organism (of strategy $w^k$).
\item   $\alpha$ as the proportion of the population that will take an action at a time step $t\rightarrow t+1$; $0<\alpha<1$.
\end{itemize}

Once the above elements $K,S,A,T,\alpha$ and initial population $P_0$ are specified - the game's process is fully specified.\\
The game's process consists of stages: The organisms in the population of a strategy $w^k$ have population distribution across states given by $P_t$.  $\alpha$ of the those individuals have $w^k$ strategy which determines the distribution of actions taken by them.  The total actions taken by all strategies determines the total transmissions among the states - thus updating $P_t$ to $P_{t+1}$. The process is embedded as Algorithm \ref{al:1}.

\begin{algorithm}[h]
\caption{Forward Stepping Algorithm}\label{al:1}
\begin{algorithmic}[1]
\setstretch{1.8}

\Procedure{Simulate}{$K,S,W,T,P_0,\alpha,t_{max}$}
\State $t\gets 0$
\While{$t<t_{max}$}
    \State $P^*_{t,k,s,a} \gets \sum_{w^k\in W^k}P_{t,k,s,w^k}w^k_{a,s}$\Comment{calculate reduced population distribution}
    \For{$k \in K$}\Comment{for each species:}
        \For{$w^k \in W^k$}\Comment{for each strategy:}
            \State $M^{t,k,w^k}_{s_1,s_2} \gets \sum_{a\in A_{k,s_2}}T_{k,s_1,a}(P^*_t) w^k_{a,s_2}$\Comment{calculate total transmissions}
            \State $z_{s_1} \gets \sum_{s\in S^k} M^{t,k,w^k}_{s_1,s}P_{t,k,s,w^k}$\Comment{apply matrix to the strategy's population}
            \State $P_{t+1,k,s,w^k} \gets \alpha z_s + (1-\alpha)P_{t,k,s,w^k}$\Comment{incorporate new population by $\alpha$}
        \EndFor
    \EndFor
    \State $t\gets t+1$
\EndWhile
\State \textbf{return} $P^*_{t_{max}}$
\EndProcedure

\end{algorithmic}
\end{algorithm}
From Algorithm \ref{al:1} is noticed that if the states were indexed $S_k=\{s_{k,0},s_{k,1},\dots\}$, that every strategy $w^k$ would have its own transmission matrix analogous to those given in section \ref{sec:2}:\begin{equation}\label{eq:transmission_matrix}m_{l,j} = M^{t,k,w^k}_{s_{k,l},s_{k,j}} = \sum_{a\in A_{k,s_j}}T_{k,s_{k,l},a}(P^*_t) w^k_{a,s_{k,j}}\end{equation}
Such that $m_{l,j}$ would be net transmission from the $j$th state to the $l$th state for the individuals that take actions. We term such a matrix the "strategy's transmission matrix".
Furthermore that if the actions were indexed $A_{k,s_{k,j}}=\{a_{k,j,0},a_{k,j,1},\dots\}$ that the probabilities of any strategy $w^k\in W^k$ would form an indexed set of numbers which we define to be the strategy's "terms".
\begin{equation}\label{eq:probability_indices}q_{j,i}=w^k_{a_{k,j,i},s_{k,j}}\end{equation}
Any appropriately dimensioned indexed set of numbers $q_{j,i}$ can be the terms of a strategy if it is "implementable". which is iff: 
$\forall j~\sum_iq_{j,i}=1$ and $\forall i,j~q_{i,j}\in\mathbb{R}_+\bigcup\{0\}$, ie. if the numbers could be taken to be probabilities of a strategy.
The terms of a strategy are the same size and dimensions as all the strategies of the same species, and so it possible to add, subtract and multiply them together element-wise to form linear combinations of strategy terms.
The result of a linear combination of strategy terms is implementable if the coefficients of the linear combination are positive and sum to unity.
In the next section we will talk of linear combinations of strategies in this manner.
\\

In anticipation of Appendix \ref{appendix5} we will present it here that: any strategy's transmission matrix has a form where it has columns are linear combination of column vectors weighted by its terms. Consider that if we index the $T$ functions as $y_{j,i,l}=T_{k,s_{k,l},a_{k,j,i}}$, and if $y_{j,i}$ denotes a column of such terms then $m_{l,j}$ has form:
\begin{equation}\label{eq:columns} m_{l,j} = \sum_iy_{j,i,l}q_{j,i} = \left[\begin{array}{c|c|c}
y_{0,0}q_{0,0}+y_{0,1}q_{0,1}+y_{0,2}q_{0,2}+\dots & 
y_{1,0}q_{1,0}+y_{1,1}q_{1,1}+\dots & 
y_{2,0}q_{2,0}+\dots
\end{array}\right]\end{equation}
